% -----------------------------------------
% OTROS PAQUETES E INSTRUCCIONES ESPECIALES
% -----------------------------------------

% PAQUETES
%%%%%%%%%%

% Para insertar PDFs
\usepackage{pdfpages}
% Para insertar código fuente estilizado
\usepackage{listings}
	\lstset{basicstyle=\ttfamily,
    		breaklines=true,
            numbers=left, 
    		numberstyle=\tiny, 
    		stepnumber=1, 
    		numbersep=6pt
            }

% Para adjuntar pdf
\usepackage{pdfpages}

% Para usar múltiples columnas (TABLAS)
\usepackage{multicol}
\usepackage{multirow}
\usepackage{colortbl}
\usepackage{bigstrut}

% Para crear árboles conceptuales
\usepackage{forest}

% Para insertar símbolos extraños
\usepackage{marvosym}

% Para insertar texto fútil
\usepackage{lipsum}

% NUEVAS INSTRUCCIONES
%%%%%%%%%%%%%%%%%%%%%%

\newcommand{\EIEx}{\textsc{Escuela \Lightning~ Ingeniería Eléctrica}}

% Definición de algunos símbolos matemáticos
\newcommand{\me}{\mathrm{e}}
\newcommand{\mi}{\mathrm{i}}
\newcommand{\mj}{\mathrm{j}}
\newcommand{\md}{\mathrm{d}}

% Listas con menos espacio entre ítemes (más ajustado)
\newcommand{\ajustado}{\itemsep0pt\parskip0pt\parsep0pt}

% FORMATO PARA EL REGLAMENTO
%%%%%%%%%%%%%%%%%%%%%%%%%%%%

\newcounter{articulo}
\newcommand{\articulo}[2]{
	{
    \stepcounter{articulo}
    \noindent
	\textbf{Artículo \thearticulo} --- \textit{#1}
	\vspace{2mm}\\
	}{
	#2
	\vspace{2mm} \\
	}
}

\newcounter{capitulo}
\newcommand{\capitulo}[1]{
	{
    \stepcounter{capitulo}
    \centering\large\bfseries
    Capítulo \Roman{capitulo}. #1
    \vspace{4mm}\\
    }
}

\hypersetup{
    pdfauthor={GrupoX},
    pdftitle={Titulo del proyecto},
    pdfsubject={Informe Final},
    pdfkeywords={PyDE, Palabra clave 2, Palabra clave 3},
    pdfproducer={LaTeX},
    pdfcreator={pdflatex}
}

